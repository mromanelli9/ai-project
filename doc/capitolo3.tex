% MICROSOFT
\section{Microsoft Cognitive Services: Computer Vision API}
Per quanto concerne il riconoscimento delle immagini, Microsoft offre le \textit{Computer Vision API} \cite{microsoft-api}, che spaziano dal riconoscimento dei volti all'analisi delle caratteristiche cromatiche dell'immagine.

\subsection{Prerequisiti}
Caratteristiche immagini:
\begin{itemize}
\item Metodo: dati grezzi (stream application/octet) o url.
\item Formati supportati: JPEG, PNG, GIF, BMP.
\item Caratteristiche minime: 50x50 pixel. 
\item Dimensione massima: 4 MB.
\item Massimo numero di immagini per collezione: un milione.
\end{itemize}
Altro:
\begin{itemize}
\item Disponibilità: Stati Uniti occidentali.
\end{itemize}


\subsection{Analisi}
In base allo scopo per cui si vuole analizzare l'immagine, il servizio mette a disposizione diversi metodi per ottenere le informazioni desiderate.

\paragraph{Tagging} Le API ritornano un insieme di etichette (in formato JSON) che descrivono gli oggetti presenti nell'immagine, come oggetti, esseri viventi, azioni, paesaggi; per ogni etichetta viene anche fornito il livello di \textit{confidence} (affidabilità). I tag non sono in alcun modo organizzati fra loro e non esiste nessun tipo di ereditarietà.
Nel caso un tag sia ambiguo viene fornito in aggiunta un \textit{hint} che ne spiega il contenuto.
Al momento la sola lingua supportata è l'inglese.

\paragraph{Classificazione} L'immagine viene classificata in categorie che seguono una tassonomia con ereditarietà di tipo padre-figlio. Questa tassonomia prevede 86 categorie\footnote{\url{https://www.microsoft.com/cognitive-services/en-us/Computer-Vision-API/documentation/Category-Taxonomy}} e classifica gli elementi visivi in modo più o meno specifico.

\paragraph{Identificazione del tipo} E' possibile classificare l'immagine come in bianco o nero o a colori, se è un disegno o se è del tipo \textit{clip-art}; in quest'ultimo caso viene fornito un livello di qualità dell'immagine, compreso fra 0 e 3.

\paragraph{Riconoscimento volti} Riconosce i volti umani e restituisce la posizione (coordinate) di questi all'interno dell'immagine, come anche età e sesso della persona.

\paragraph{Contenuto personalizzato} Ideato per raffinare la tassonomia a 86 categorie utilizzando informazioni specifiche sul dominio. Attualmente è supportato solamente il riconoscimento dei volti delle persone famose.

\paragraph{Generazione di descrizioni} Genera una lista di frasi (in lingua inglese) che descrivono il contenuto dell'immagine, ordinate secondo un livello di affidabilità calcolato per ogni descrizione.

\paragraph{Estrazione colori} Identifica i colori analizzandoli in tre contesti: di sfondo, in primo piano e d'insieme; i colori sono raggruppati in 12 colori predominanti. Classifica le immagini fra in bianco e nero e a colori.

\paragraph{Riconoscimento contenuti non adatti ai minori} Riconosce materiali pornografici e contenuti osé in generale. Può essere impostato un livello per il filtro.

\paragraph{Riconoscimento del testo (OCR)} Rileva il testo presente nell'immagine e lo trasforma in un flusso di parole, ruota l'immagine se necessario per rendere il testo orizzontale e fornisce le coordinate per ogni parola. Al momento sono supportati 21 linguaggi, fra cui l'inglese, l'italiano, il francese, il tedesco e lo spagnolo.

L'accuratezza del riconoscimento dipende dalla qualità dell'immagine ed eventuali errori possono essere causati da immagini sfuocate, scrittura a mano, testo troppo piccolo, ecc.
   
\paragraph{Creazione anteprime} Un'anteprima è una rappresentazione dell'immagine in scala ridotta. L'immagine viene prima analizzata e poi ritagliata secondo la ``regione di interesse'' (ROI); il rapporto dell'immagine (\textit{aspect ratio}) può essere impostato secondo le proprie preferenze.

\subsection{Tariffe} Due tipologie di piani:
\begin{itemize}
\item Gratuito: fino a 5000 chiamate al mese, massimo 20 chiamate al minuto;
\item Standard: 0,015\$ a chiamata, fino a 10 TPS.
\end{itemize}
