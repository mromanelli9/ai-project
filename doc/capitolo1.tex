% INTRODUZIONE
\section{Introduzione}

\subsection{Scopo del progetto}
Il progetto consiste in un'analisi delle principali piattaforme che offrono servizi di \textit{Cognitive Computing},
partendo con una panoramica generale per poi focalizzarsi sull'analisi delle immagini.

\subsection{La teoria computazionale della mente}
Con il termine \textit{Cognitive Computing} si indicano quelle piattaforme di calcolo che comprendo al loro interno servizi
di \textit{machine learning}, ragionamento, analisi del linguaggio naturale, del testo, delle immagini, con lo scopo di simulare
il funzionamento del cervello umano e cercare di migliorare i processi di decisione.
Sviluppare, quindi, un meccanismo unico e universale ispirato alle potenzialità della nostra mente.
Piuttosto che comporre diverse soluzioni separate una dall'altra, si cerca di implementare una teoria computazionale unificata della mente umana.
Allen Newell, un pioniere nell'intelligenza artificiale, in \cite{newell92} la definì come:
\begin{quote}
% a single set of mechanisms for all of cognitive behavior. Our ultimate goal is a unified theory of human cognition
Un insieme unico di meccanismi che include tutti i comportamenti cognitivi. L'obbiettivo finale è una teoria unificata del comportamento umano.
\end{quote}

\subsection{I servizi cognitivi}
% TODO

\subsection*{}
L'articolo procederà come segue: nel Capitolo 2 verranno presentate brevemente le piattaforme prese in esame, illustrando per ognuna i vari sevizi offerti.
Seguiranno poi nei capitoli 3-6 le analisi dettagliate di ciascuna piattaforma, focalizzandosi sull'analisi delle immagini.
Il Capitolo 7 riassume in forma sintetica le funzionalità viste.
Il Capitolo 8 concluderà con le conclusioni dell'autore e alcuni possibili sviluppi di questo lavoro.
