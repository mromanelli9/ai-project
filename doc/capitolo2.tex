% PANORAMICA
\section{Panoramica}
Molte fra le maggiori aziende nel settore tecnologico e informatico hanno cominciato a offrire servizi cognitivi.
% Da migliorare
%
Fra queste ne sono state individuate quattro, che offrono un'ampia gamma di sevizi, tariffe diversificate in base all'esigenza e la possibilità di sopportare carichi di lavoro molto ingenti.
I servizi presi in esame saranno, quindi:
\begin{itemize}
\item Microsoft Cognitive Services \cite{mcs-link} (Microsoft Corporation),
\item Watson Services (Bluemix) \cite{ibm-link} (IBM: International Business Machines Corporation),
\item Amazon Artificial Intelligence \cite{amazon-link} (Amazon.com, Inc)
\item Google Cloud Machine Learning Services \cite{google-link} (Google Inc.)
\end{itemize}

Ognuna di queste suddivide i servizi in macro aeree che possono essere riassunte così: visione artificiale, sintesi vocale, linguaggio naturale, ricerca, tecniche di apprendimento e altro.
Non tutte le piattaforme utilizzando la stessa nomenclatura e i servivi al loro interno possono variare leggermente, ma le aree di interesse coperte sono più o meno queste.
L'area denominata \textit{visione artificiale} include riconoscimento visivo di immagini e video, estrazione di informazioni, riconoscimento volti ed emozioni.
Con \textit{sintesi vocale} vengono indicati quei servizi atti all'elaborazione dell'audio e alla sua trasformazione in testo o in strutture dati adatte all'analisi.
L'area \textit{linguaggio} permette di analizzare ed elaborare il linguaggio naturale, come ad esempio comprendere comandi ed estrapolare informazioni importanti da un dato contesto.
La \textit{ricerca} permette di sfruttare le potenzialità del motore di ricerca offerto dalla compagnia stessa (se presente) o di eseguire ricerche avanzate all'interno di collezioni creati appositamente.
\textit{Tecniche di apprendimento} permette la fruizione di algoritmi e modelli, oltre che alla loro creazione, per l'apprendimento automatico ed approfondito.
La macro area \textit{altro} comprende tutti quei servizi che non ricadono nelle aree precedentemente descritte, come ad esempio sistemi di raccomandazione o altro.
Infine, la tabella \ref{tab-macro-aree} riassume con i servivi offerti (ad alto livello) da ogni piattaforma, inseriti nelle macro aree di riferimento.

E' necessario aggiungere, tuttavia, che questa classificazione vuole fornire solamente una visone generale e grandi linee dei servizi disponibili e non vuole essere esaustiva.
Per l'offerta completa si rimanda alle rispettive documentazioni. Inoltre, l'assenza di voci in una macro area per una certa piattaforma non esclude la presenza di relativi servizi; potrebbero essere, infatti, presenti sotto altri nomi, piattaforme, framework o comunque presenti negli altri servizi.
%
%
\begin{table}[!h]
\centering
{\scriptsize
\begin{tabularx}{.9\textwidth}{l|X|X|X|X}
\toprule
Aree & Microsoft Corporation & IBM & Amazon.com, Inc & Google Inc.\\ \hline
\midrule                           
\multicolumn{1}{l|}{Visione artificiale}
& Computer Vision API, Content Moderator, Emotion API, Face API, Video API
& Visual Recognition
& Amazon Rekognition
& Video Intelligence API, Vision API \\ \hline
\multicolumn{1}{l|}{Sintesi vocale}
& Bing Speech API, Custom Speech Service, Speaker Recognition API
& Speech to Text, Text To Speech
& Amazon Polly
& Speech API \\ \hline
\multicolumn{1}{l|}{Linguaggio naturale}
& Bing Spell Check API, Language Understanding Intelligent Service, Linguistic Analysis API, Text Analytics API, Translator API, Web Language Model API
& AlchemyLanguage, Conversation, Dialog, Document Conversion, Language Translator, Natural Language Classifier, Natural Language Understanding, Personality Insights, Retrieve and Rank, Tone Analyzer
& Amazon Lex
& Natural Language API, Translation API \\ \hline
\multicolumn{1}{l|}{Ricerca}
& Bing Autosuggest API, Bing Image Search API, Bing News Search API, Bing Video Search API, Bing Web Search API, Academic Knowledge API, Knowledge Exploration Service
& Discovery, Discovery News, 
& -
& - \\ \hline
\multicolumn{1}{l|}{Apprendimento}
& -
& -
& Amazon Machine Learning, Apache Spark su Amazon EMR
& Machine Learning Engine \\ \hline
\multicolumn{1}{l|}{Altro}
& Entity Linking Intelligence Service, QnA Maker, Recommendations API
& Tradeoff Analytics
& -
& Jobs API \\ \hline
\end{tabularx}}
\caption{Tabella riassuntiva dei servizi offerti, raggruppati per macro aree}
\label{tab-macro-aree}
\end{table}
%