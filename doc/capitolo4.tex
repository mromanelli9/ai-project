% IBM
\section{IBM Watson Services: Visual Recognition}
Il servizio di Visual Recognition\cite{ibm-api} utilizza tecniche e algoritmi di \textit{deep learning} per identificare scene, oggetti, visi di persone nell'immagine che viene fornita come input al servizio. Permette, inoltre, la creazione e l'addestramento di un classificatore personalizzato per l'identificazione di elementi in base alle necessità dello sviluppatore.

\paragraph{Classificazione} Per ogni immagine sottoposta a classificazione viene fornito in riposa una lista di coppie classe-punteggio per ogni classificatore selezionato. Il punteggio è compreso in un intervallo $0-1$, dove un valore maggiore indica una probabilità più alta che la classe descriva l'immagine; la soglia di default perché un valore sia ritornato da un classificatore è $0,5$.
Le classi sono organizzate in categorie e sotto-categorie dove il livello più astratto comprende categorie quali animali, persone, cibo, sport, natura, eccetera.

Le lingue sopportate\footnote{Al momento della stesura di questo documento.} nella risposta sono l'inglese, spagnolo, arabo o giapponese. 

\paragraph{Riconoscimento dei volti} Analizza i volti presenti l'immagine e ne deriva alcune informazioni, come età stimata, sesso o nome del personaggio famoso (nel caso ci sia). Anche in questo caso viene fornito un punteggio (nell'intervallo $0-1$) atto ad indicare una maggiore probabilità di correlazione.

\paragraph{Classificatore personalizzato} Permette di creare un nuovo classificatore e di addestrarlo su un dato insieme di immagini. Queste sono inviate in un file compresso e devono comprendere o due immagini d'esempio positive o una positiva e una negativa. L'insieme contente le immagini d'esempio positive serve a creare le classi che definiscono il nuovo classificatore. Il complementare definisce invece quello che il classificatore \textit{non} deve essere; le immagini d'esempio negative non devono contente i soggetti presenti nelle immagini positive.

Se, ad esempio, si volesse creare un classificatore ``frutta'' si potrebbe utilizzare un file compresso contente immagini di pere, uno contente immagini di mele e uno con immagini di banane.
Per le immagini d'esempio negative si potrebbero utilizzare immagini di verdure.

\paragraph{Collezioni} Questa funzione\footnote{Questa funzione è ancora in fase BETA} permette di creare una nuova collezione, aggiungere immagini a questa e utilizzare la \textit{Similarity Search} per cercare immagini simile all'interno della collezione.

\paragraph{Note per la privacy} Per default, tutte le immagini e le informazioni inviate vengono salvate e utilizzate per migliorare il servizio. Per evitare questo è necessario impostare diversamente il parametro \textsf{X-Watson-Learning-Opt-Out} in ogni richiesta inviata.

\subsection{Tariffe}
Il piano gratuito prevede la possibilità di:
\begin{enumerate}
\item classificare 250 immagini al giorno,
\item addestrare un solo classificatore personalizzato con massimo 5000 immagini.
\end{enumerate}
Il piano \textit{standard} prevede:
\begin{enumerate}
\item per la classificazione: $0,002$ dollari a immagine,
\item per il riconoscimento volti: $0,004$ dollari a immagine,
\item per l'addestramento classificatore: $0,10$ dollari a immagine,
\item per la classificazione con classificatore personalizzato: $0,004$ dollari a immagine.
\end{enumerate}


