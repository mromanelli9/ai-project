% !TEX encoding = UTF-8
% !TEX program = pdflatex
% !TEX root = relazione.tex
% !TeX spellcheck = it_IT

% INTRODUZIONE
\section{Introduzione}\label{sec:introduzione}
\subsection{Scopo dell'analisi}\label{subsec:scopo-analisi}
Questa analisi consiste in una panoramica sulle principali piattaforme che offrono servizi di \textit{Cognitive Computing},
partendo con un'introduzione generale per poi focalizzarsi sull'analisi delle immagini.
%
\subsection{I servizi cognitivi}\label{subsec:i-servizi-cognitivi}
Si tratta di una serie di servizi che consentono agli sviluppatori di realizzare applicazioni in grado di analizzare e interpretare la realtà,
usando quelli che si usa chiamare col nome di Metodi di Comunicazione Naturale.
Questi servizi spaziano in aree come la visione, il linguaggio, il parlato, la ricerca, la conoscenza, eccetera;
per esempio, l'area di visione racchiude quelle tecniche per l'analisi, la rappresentazione di immagini e video, mentre
nell'area del linguaggio fornisce strumenti per capire meglio cosa vuole l'utente tramite l'interpretazione della scrittura o del linguaggio naturale.

Dal punto di vista tecnico, i Servizi Cognitivi sono una raccolta di metodi (API) che aiutano professionisti e programmatori
a realizzare programmi ``intelligenti''.
Sostanzialmente, si effettuano delle chiamate \textit{REST} (a \textit{endpoint} http)
passando le immagini o i dati da analizzare e il servizio restituirà un risulato (a seconda del metodo utilizzato).
Questo è possibile sfruttando le complesse infrastutture cloud e lo stato dell'arte di tecniche e algoritmi.
%
\paragraph*{}
L'articolo procederà come segue: nella Sezione~\ref{sec:panoramica} verranno presentate brevemente le piattaforme prese in esame, illustrando per ognuna i vari sevizi offerti.
Seguiranno poi nelle sezioni 3-6 le analisi dettagliate di ciascuna piattaforma, focalizzandosi sull'analisi delle immagini.
La Sezione~\ref{sec:esempi} verranno presentati alcuni esempi di applicazione, mentre
nella Sezione~\ref{sec:applicazioni-reali} saranno prese in esame situazioni reali valutando l'aspetto finanziario.
Infine, la Sezione~\ref{sec:conclusioni} riassumerà brevemente i concetti visti inserendo alcune conclusioni finali e possibili sviluppi futuri.
In Appendice~\ref{sec:tabelle-riassuntive} sono presenti alcune tabelle riassuntive.
