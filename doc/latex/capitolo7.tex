% !TEX encoding = UTF-8
% !TEX program = pdflatex
% !TEX root = relazione.tex
% !TeX spellcheck = it_IT

% ESEMPI
\section{Esempi}\label{sec:esempi}
Parlando di riconoscimenti visivo, è naturale inserire anche alcuni esempi su immagini di prova per mettere a confronto
le diverse piattaforme nei casi specifici.
Sono state identificate quattro funzionalità presenti in tutti i servizi (come si può vedere dalla tabella~\ref{tab:riass-funzionalita} in appendice):
riconoscimento di oggetti specifici, il riconscimento di scene e il riconscimento di volti.

Per consentire la maggior imparzialità, le immagini sono state selezionate casualmente da alcuni motori di ricerca.
%
\subsection{Riconoscimento di un oggetto specifico}\label{subsec:riconscimento-oggetto-specifico}
L'immagine presa come riferimento è la figura~\ref{fig:scontrino} che raffigura uno scontrino fiscale.
\begin{figure}[!h]
\begin{center}
	\includegraphics[scale=.5]{scontrino.png}
{\scriptsize \caption{Immagine usata come riferimento.}
\label{fig:scontrino}}
\end{center}
\end{figure}
%
\paragraph{Microsoft} Utilizzando le Computer Vision API con i metodi di tagging, classificazione e generazione di descrizioni,
il risultato ottenuto è\footnote{Alcuni campi, non direttamente rilevanti, sono stati omessi per faciltare la lettura.}:
%
\begin{lstlisting}[style=myJSON, caption=Risultato dell'interrogazione con le Computer Vision API., label=lst:risultati-microsoft-scontrino]
{
   'categories':[{
         'name':'text_menu',
         'score':0.85546875
    }],
   'description':{
      'captions':[{
            'confidence':0.6941834877564524,
            'text':'a close up of a receipt'
         }],
      'tags':['text', 'receipt']
   },
   'tags':[
      {'confidence':0.9944400787353516,
         'name':'text'},
      {'confidence':0.9528059959411621,
         'name':'receipt'}]
}
\end{lstlisting}
%
Come si può vedere, l'immagine viene classificata come \textsf{menu} nella categoria \textsf{text} (della macro-categoria \textsf{object});
considerando che in tutto ci sono meno di 90 categorie (per la tassonomia vedere la sezione~\ref{subsec:computer-vision-api}), il risultato è piuttosto soddisfacente.
Descrive l'immagine come: ``Un primo piano di uno scontrino'', quindi è decisamente esatta.
Infine, fornisce l'etichetta \textsf{testo} con un livello di affidabilità del $99,44\%$ e \textsf{scontrino} con il $95,28\%$.
Riassumendo, il sistema riconosce perfettamente l'oggetto nel suo complesso, anche se non ci fornisce particolari dettagli.
%%
\subsection{Riconoscimento di volti}\label{subsec:riconscimento-volti}
In questo caso, l'immagine~\ref{fig:stallone} raffigura due personaggi famosi. Questo forse potrebbe facilitare il riconoscimento
considerando che una caratteristica offerta (in tutte le piattaforme) è il riconoscimento di persone famose.
%
\begin{figure}[!h]
\begin{center}
	\includegraphics[scale=.5]{Sylvester-Stallone-making-it-rain-on-pal-Arnold-Schwarzenegger.jpg}
{\scriptsize \caption{Immagine usata come riferimento per il riconscimento di volti.}
\label{fig:stallone}}
\end{center}
\end{figure}
%
\paragraph{Microsoft}In questo caso sono state utilizzate le Face API in modalità \textit{Rilevamento volti}.
\begin{figure}[!h]
\begin{center}
	\includegraphics[scale=.5]{Sylvester-Stallone-making-it-rain-on-pal-Arnold-Schwarzenegger_post.png}
{\scriptsize \caption{Immagine riferimento dove sono stati evidenziati i volti riscontrati.}
\label{fig:stallone-post}}
\end{center}
\end{figure}
\begin{lstlisting}[style=myJSON, caption=Risultato dell'interrogazione con le Face API., label=lst:risultati-microsoft-volti]
	{
	   "FaceId":"e93d3fd5-56d8-4009-b1ba-341d29520262",
	   "FaceRectangle":{ "Top":45, "Left":84, "Width":77, "Height":77 },
	   "FaceAttributes":{
	      "Hair":{
	         "Bald":0.03,
	         "Invisible":false,
	         "HairColor":[
	            { "Color":"brown",
	               "Confidence":0.8 },
	            { "Color":"gray",
	               "Confidence":0.58 }]},
	      "Smile":0.062,
	      "HeadPose":{ "Pitch":0.0, "Roll":7.9, "Yaw":2.2 },
	      "Gender":"male",
	      "Age":59.5,
	      "Glasses":"Sunglasses",
	      "Emotion":{
	         "Anger":0.001,
	         "Contempt":0.0,
	         "Disgust":0.0,
	         "Fear":0.0,
	         "Happiness":0.062,
	         "Neutral":0.901,
	         "Sadness":0.034,
	         "Surprise":0.002
	      }
	   },
	   "FaceLandmarks":{
	      "PupilLeft":{
	         "X":105.7,
	         "Y":65.0 },
	      "PupilRight":{
	         "X":143.0,
	         "Y":69.5 },
	      "NoseTip":{
	         "X":122.4,
	         "Y":84.8 },
	      "MouthLeft":{
	         "X":102.9,
	         "Y":98.1 },
	      "MouthRight":{
	         "X":136.1,
	         "Y":102.0 },
	   }
	}
\end{lstlisting}
%
Anche in questo caso sono state omesse alcune voci ed è stato riportato solamente il risultato per il primo volto (persona a sinistra).
