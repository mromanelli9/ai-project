% !TEX encoding = UTF-8
% !TEX program = pdflatex
% !TEX root = relazione.tex
% !TeX spellcheck = it_IT
%
% GOOGLE
\section{Google Cloud Machine Learning Services}
Le \textit{Google Cloud Vision API} \cite{google-api} permettono di analizzare un'immagine e classificarla in categorie, rilevare oggetti e volti, cercare parole,
moderare contenuti offensivi e molto altro. Le immagini possono essere caricate assieme la richiesta, oppure utilizzare quelle già presenti nel Google Cloud Storage.
%
\subsection{Prerequisiti}
Le immagini passate al servizio devono rispettare i seguenti requisiti:
\begin{itemize}
\item Metodo: dati grezzi (stream application/octet) o tramite Google Cloud Storage URIs.
\item Formati supportati: JPEG, PNG8, PNG24, GIF, Animated GIF\footnote{Viene considerato solo il primo frame.},
BMP, WEBP, RAW, ICO.
\item Caratteristiche minime: 640x480 pixel.
\item Dimensione massima: 4 MB.
\end{itemize}
%Altro:
%\begin{itemize}
%\item Disponibilità: %Stati Uniti orientali, Stati Uniti occidentali, Europa.
%\end{itemize}
\subsection{Cloud Vision API}
Per ogni immagine all'interno di una richiesta, è possibile specificare uno o più tipi di metodi (\textit{features}) corrispondenti alle azioni desiderate.
\paragraph{\textsf{LABEL\_DETECTION}} Rileva elementi all'interno di un'ampia gamma di categorie che spaziano da animali, trasporti, eccetera.
Per ogni elemento rilevato viene associato un punteggio che indica il grado di affidabilità che quella categoria rispecchi veramente l'elemento.
%
\paragraph{\textsf{FACE\_DETECTION}} Rileva i volti presenti nell'immagine e restituisce un insieme di metadati (per ogni volto) che includono:
\begin{itemize}
\item le coordinate di due poligoni, uno che circonda tutta la testa mentre un altro che circonda solamente il viso (la parte frontale della testa ricompera di pelle),
\item le coordinate per un insieme di punti di riferimento del viso, fra cui occhi, orecchie, sopracciglia, labbra, naso, bocca\footnote{Per la lista completa vedere \url{https://cloud.google.com/vision/docs/reference/rest/v1/images/annotate\#Landmark}.}.
\item il grado di rotazione del volto, espresso nella forma degli angoli di Tait-Bryan (imbardata, rollio e beccheggio),
\item il grado di affidabilità (che l'elemento rilevato sia effettivamente un volto),
\item il grado di affidabilità dei punti di riferimento,
\item le probabilità che il volto esprima gioia, tristezza, rabbia, sorpresa e che sia presente un cappello,
\item le probabilità che l'immagine sia sottoesposta o sfuocata.
\end{itemize}
Le probabilità sono espresse secondo una scala a sei valori: \textsf{UNKNOWN}, \textsf{VERY\_UNLIKELY}, \textsf{UNLIKELY}, \textsf{POSSIBLE}, \textsf{LIKELY} e \textsf{VERY\_LIKELY}.
%
\paragraph{\textsf{TEXT\_DETECTION}} Effettua le stesse funzioni di un OCR (Optical Character Recognition): riconosce caratteri e parole all'interno dell'immagine.
Restituisce la lingua del testo rilevato, il testo e le coordinate dei poligoni, uno per l'intera frase e uno per ogni parola che la costituisce.
%
\paragraph{\textsf{DOCUMENT\_TEXT\_DETECTION}} Esegue la stessa funzione del metodo precedente, ma è ottimizzato per immagini con molto testo.
L'oggetto restituito da questo metodo è costituito dalla struttura: \textsf{Page}$\to$\textsf{Block}$\to$\textsf{Paragraph}$\to$\textsf{Word}$\to$\textsf{Symbol}.

Una pagina contiene il linguaggio del testo al suo interno, l'altezza, la larghezza e una lista di blocchi.
Un blocco contiene la lingua, le coordinate del poligono che racchiude il blocco, il tipo di blocco e una lista di paragrafi;
i tipi sono: sconosciuto, testo, tabella, figura, linea o codice a barre.
Il paragrafo include il testo che a sua volta contiene le singole parole che contengono i simboli.
%
\paragraph{\textsf{LANDMARK\_DETECTION}} Questo metodo permette il riconoscimento di famose elementi naturali e artificiali, come ad esempio la Torre Eiffel, il ``Paris Hotel and Casino'' a Las Vegas, la Fontana di Trevi.
Il valore ritornato contiene, oltre ai soliti elementi, le coordinate latitudine/longitudine.
%
\paragraph{\textsf{LOGO\_DETECTION}} Riconosce loghi e marchi di vari prodotti comuni o famosi.
%
\paragraph{\textsf{SAFE\_SEARCH\_DETECTION}} Rileva contenti non adatti ai minori e restituisce la probabilità che l'immagine contenga
contenuti per adulti, relativi a un ambito medico, violenti o che sia una modifica di un'immagine originale con scopi ludici o offensivi.
I livelli di probabilità sono espressi secondo la scala a sei grada illustrata precedentemente.
%
\paragraph{\textsf{IMAGE\_PROPERTIES}} Restituisce una lista di colori dominanti; per ogni colore è presente la sua descrizione in formato RGB,
un valore di punteggio e il numero di pixel di quel colore presenti nell'immagine (in rapporto al totale).
%
\paragraph{\textsf{CROP\_HINTS}} Suggerisce i punti dove meglio ritagliare l'immagine.
Viene restituito un valore di \textsf{importanceFraction}, che indica il rapporto fra l'``importanza'' dell'immagine ritagliata e l'immagine originale.
%
\paragraph{\textsf{WEB\_DETECTION}} Restituisce informazioni presenti nel web rilevanti per l'immagine:
\begin{itemize}
\item \textsf{webEntities}: deduce elementi dell'immagine da immagini simili nel web;
\item \textsf{fullMatchingImages}: immagini presenti nel web molto simili a quella di partenza (spesso sono copie);
\item \textsf{partialMatchingImages}: immagini presenti nel web che presentano elementi chiave in comune con quella di partenza;
\item \textsf{webEntities}: pagine web che contengono immagine simili a quella di partenza.
\end{itemize}
%
Le funzioni \textsf{LANDMARK\_DETECTION}, \textsf{CROP\_HINTS} e \textsf{WEB\_DETECTION} sono in versione beta e potrebbero variare nel tempo.
Non è consigliabile l'utilizzo applicativo e/o in applicazione critiche.
%
\subsection{Tariffe}
La Cloud Vision API fornisce diversi metodi per analizzare un'immagine.
Ad ogni metodo corrisponde una tariffa che viene applicata ad ogni utilizzo su un'immagine, chiamato \textit{unità};
se, ad esempio, due metodi sono applicati alla stessa immagine allora ai fini tariffari verranno conteggiati separatamente.
Il prezzo è determinato dal numero di unità per mese e il costo per ogni metodo è indicato per 1000 unità/mese in dollari, come indicato in tabella \ref{google-tariffe}.
%
\begin{table}[!h]
\centering
{\footnotesize
\begin{tabularx}{\linewidth}{l?c|c|c|c}
\toprule
Metodo & 1-1K unità/mese & 1K-1M unità/mese & 1M-5M unità/mese & 5M-20M unità/mese \\ \hline
\midrule
\multicolumn{1}{l?}{\textsf{Label Detection}} & Gratis & 1,50 & 1,50 & 1,00 \\ \hline
\multicolumn{1}{l?}{\textsf{OCR}} & Gratis & 1,50 & 1,50 & 0,60 \\ \hline
\multicolumn{1}{l?}{\textsf{Explicit Content Detection}} & Gratis & 1,50\footnote{\label{explicit}Gratis se utilizzato assieme a \textsf{Label Detection}}
& 1,50\textsuperscript{\ref{explicit}} & 0,60\textsuperscript{\ref{explicit}} \\ \hline
\multicolumn{1}{l?}{\textsf{Facial Detection}} & Gratis & 1,50 & 1,50 & 0,60 \\ \hline
\multicolumn{1}{l?}{\textsf{Landmark Detection}} & Gratis & 1,50 & 1,50 & 0,60 \\ \hline
\multicolumn{1}{l?}{\textsf{Logo Detection}} & Gratis & 1,50 & 1,50 & 0,60 \\ \hline
\multicolumn{1}{l?}{\textsf{Image Properties}} & Gratis & 1,50 & 1,50 & 0,60 \\ \hline
\multicolumn{1}{l?}{\textsf{Web Detection}} & Gratis & 3,50 & 3,50 & n.d.\footnote{\label{explicit} Contattare Google per maggiori informazioni.} \\ \hline
\multicolumn{1}{l?}{\textsf{Web Detection}} & Gratis & 3,50 & 3,50 & n.d.\textsuperscript{\ref{explicit}} \\ \hline
\end{tabularx}}
\caption{Tariffe per la Cloud Vision API}
\label{tab:google-tariffe}
\end{table}
