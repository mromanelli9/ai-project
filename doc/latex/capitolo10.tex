% !TEX encoding = UTF-8
% !TEX program = pdflatex
% !TEX root = relazione.tex
% !TeX spellcheck = it_IT

% CONCLUSIONI
\section{Conclusioni}\label{sec:conclusioni}
In questo lavoro è stato possibile osservare le caratteristiche salienti dell'analisi visiva nell'ambito
dei servizi cognitivi e le possibilità offerte da alcuni fra i maggiori fornitori di queste.
Grazie a questo è possibile, quindi, conoscere i progressi effettuati fino a questo punto,
notare le mancanze che andranno colmate e verso dove si stanno muovendo i prossimi passi.

È opinione dell'autore che i servizi disponibili a tutt'oggi (perlomeno quelli analizzati)
diano buoni, e in alcuni casi anche ottimi, risultati e che permettano di conseguenza
la realizzazione di applicazioni in grado di ineragire con l'ambiente reale con un buon grado di affidabilità
(per sistemi non critici).
%
\subsection{Sviluppi futuri}
Inizialmente si potrebbe includere anche l'analisi di video (che non è stata coperta in questo lavoro).
Il passo successivo sarebbe sicuramente lo studio delle altre macro-aree, come per esempio quella del linguaggio o della ricerca.
Nonostante siano stati inclusi alcuni esempi di utilizzo della API, questi sono molto generici
e danno solo un'idea di massima dei servizi.
Si potrebbe, quindi, includere lo studio di un'applicazione reale (o il più possibile reale), in modo da osservare
il comportamente delle API con un bisogno e problema effettivo.
