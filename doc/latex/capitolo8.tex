% !TEX encoding = UTF-8
% !TEX program = pdflatex
% !TEX root = relazione.tex
% !TeX spellcheck = it_IT

% DETTAGLI TECNICI
\section{Dettagli tecnici}\label{sec:dettagli-tecnici}
Finora si è parlato a livello astratto di servizi, piattaforme e metodi, senza spiegare esattamente come avviene
una chiamata a una certa API, quali sono i valori restituti e come elaborarli.
Verranno presi in considerazione anche le problematiche legate alla scelta del linguaggio di programmazione
da utilizzare e/o la presenza di eventuali framework disponibli.
%
\subsection{REST API}\label{subsec:rest-api}
Per accedere alle funzionalità è sufficiente effettuare una chiamata (REST: \textit{REpresentational State Transfer})
all'indirizzo HTTP (\textit{endpoint}) fornito nelle documentazioni.

Sono necessari, quindi, i seguenti passi:
\begin{enumerate}
	\item Inizializzazione:
	\begin{enumerate}
		\item Preparazione dei parametri
		\item inizializzazione della chiamata
	\end{enumerate}
	\item Chiamata all'endpoint
	\item Elaborazione del risultato
\end{enumerate}

Si prenda ora, a titolo d'esempio, una chiamata alle Microsoft Computer Vision API
per l'analisi di un'immagine \textsf{ImageRawData} utilizzando il metodo \textsf{analyses}
(permette di estrarre le caratteristiche visive dall'immagine) per avere una descizione
dell'immagine.
%
\paragraph{Inizializzazione}
Innanzitutto è doveroso far notare che, per poter utilizzare le API, è necessario
essere in possesso di un (o più) codice di autenticazione.
Quindi:
%
\begin{lstlisting}[language=Python,style=mystyle]
endpoint = 'https://westcentralus.api.cognitive.microsoft.com/vision/v1/analyses'
key = # Inserire qui la primary key
params = {
	"visualFeatures" : "Description",
	"language" : "en"
}
headers["Ocp-Apim-Subscription-Key"] = _key
headers["Content-Type"] = "application/octet-stream"
\end{lstlisting}
%
\paragraph{Chiamata all'endpoint}
Si effettua ora una chiamata tramite un certo metodo \textsf{httpRequest} adibito a
effettuare chiamate HTTP:
%
\begin{lstlisting}[language=Python,style=mystyle]
res = httpRequest( endpoint, json=None, ImageRawData, headers, params )
\end{lstlisting}
In questo caso specifichiamo \textsf{json=None} in quanto si carica direttamente l'immagine
invece che passare un indirizzo URL.
%
\paragraph{Elaborazione del risultato}
Se la richiesta va a buon fine (si è autorizzati, c'è connessione, la richiesta è conforme,\ldots)
si otterranno i risultati del metodo desiderato in formato JSON, come ad esempio:
\begin{lstlisting}[style=myJSON]
{ 'description':
	{ 'captions':
		[ { 'confidence': 0.8454074008443188,
						'text': 'a large airplane at an airport'}
		]}
}
\end{lstlisting}
%
\subsection{Scelta del linguaggio di programmazione}\label{subsec:scelta-linguaggio}
Questa è una scelta importante che va ponderata attentamente.
Potrebbe essere il caso che una piattaforma non offra la documentazione necessaria
per un certo linguaggio oppure che per alcuni ci siano framework o librerie
già scritte, facilitando il lavoro dello sviluppatore.
In Tabella~\ref{tab:ling-programazione} sono stati alcuni dei linguaggi
di programmazione previsti nella documentazione ufficiale\footnote{L'assenza di un determinato linguaggio
all'interno della tabella non preclude l'utilizzo dello stesso per i servizi visti.}.

Se nella documentazione sono presenti esempi con un certo linguaggio, allora questo sarà presente
nella prima riga della tabella.
La seconda riga, invece, evidenzia la presenza di liberie/framework per un certo linguaggio.
%
\begin{table}
\centering
\caption{Linguaggi di programmazione presenti nelle documentazioni ufficiali.}
\label{tab:ling-programazione}
{\tiny
\begin{tabularx}{\linewidth}{c?l|l|l|l}
\toprule
	& Microsoft Vision & IBM Visual Recognition & Amazon Rekognition     & Google Cloud Vision \\ \hline
\midrule
Esempi forniti                & cURL             & cURL                   & AWS CLI             & cURL \\
	& C\#              & Python                 & Java                   & C\#                 \\
	& Java             & Java                   &                        & Java                \\
	& Javascript       & Node.js                &                        & GO                  \\
	& Object C         &                        &                        & Node.js             \\
	& PHP              &                        &                        & PHP                 \\
	& Python           &                        &                        & Python              \\
	& Ruby             &                        &                        & Ruby                \\ \hline
Framework                     & n.p.             & Node.js                & Command Line Interface & tutti quelli sopra  \\
	&                  & Swift                  & Javascript             &                     \\
	&                  & Python                 & Python                 &                     \\
	&                  & Java                   & Java                   &                     \\
	&                  & Unity                  & Ruby                   &                     \\
	&                  & .NET                   & .NET                   &                     \\
\end{tabularx}}
\end{table}
%
