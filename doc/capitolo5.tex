% AMAZON
\section{Amazon Artificial Intelligence: Amazon Rekognition}
Amazon Rekognition \cite{amazon-api} è il servizio di Amazon che permette di riconoscere oggetti, volti, scene, nelle immagini e molto altro.
Inoltre permette l'integrazione con le altre piattaforme offerte da Amazon, come Amazon S3, AWS Lambda e altri servizi AWS.

\subsection{Prerequisiti}
Caratteristiche immagini:
\begin{itemize}
\item Metodo: dati grezzi (stream application/octet) o oggetto Amazon S3.
\item Formati supportati: PNG, JPEG.
\item Caratteristiche minime: 80x80 pixel. 
\item Dimensione massima: 5 MB (dati grezzi), 15 MB (oggetto Amazon S3).
\item Massimo numero di immagini per collezione: un milione.
\end{itemize}
Altro:
\begin{itemize}
\item Disponibilità: Stati Uniti orientali, Stati Uniti occidentali, Europa.
\end{itemize}

\subsection{Analisi}
Prima di descrivere cosa si può fare o meno con Amazon Rekognition, è necessaria fare una distinzione; le operazioni fornite dal servizio si suddividono in due categorie:
\begin{itemize}
\item Operazioni volatili (Non-storage API operations): le operazioni in questo gruppo non salvano alcuna informazione sui server Amazon.
\item Operazioni persistenti (Storage-based API operations): l'utilizzo di queste operazioni comporta il salvataggio di alcuni metadati sui server.
Ad esempio nella ricerca di un volto viene salvata la rappresentazione vettoriale dei volti rilevati (mentre l'immagine di partenza no).
\end{itemize}

\paragraph{Rilevamento scene e oggetti} Permette di rilevare automaticamente oggetti, come ad esempio veicoli, alberi, animali e con il relativo punteggio
(\textit{confidence score}) che ne indica il grado di affidabilità (probabilità che sia corretto).
Permette, inoltre, il riconoscimento di scene (una spiaggia o un tramonto), eventi (matrimoni, feste di compleanno) e concetti astratti (serata, paesaggio, natura).
%Questa operazione (\textit{DetectLabels}) non salva alcuna informazione sul server.

\paragraph{Analisi volti} Permette il riconoscimento di volti all'interno dell'immagine, la loro localizzazione spaziale e l'analisi di attributi facciali come
il sesso, l'eta, emozioni, se la persona sta sorridendo, se ha gli occhi aperti, presenza o assenza di barba/baffi, eccetera
\footnote{Per una lista esaustiva si faccia riferimento alla documentazione ufficiale all'indirizzo: \url{http://docs.aws.amazon.com/rekognition/latest/dg/API_Types.html}.}.
La localizzazione avviene tramite un immaginario rettangolo (sotto forma di coordinate $(x, y)$ degli angoli) che circonda ogni viso rilevato e dei punti di rifermento su elementi come il naso,
gli occhi, le orecchie, la bocca, ecc.
Naturalmente, gli algoritmi di riconoscimento sono più affidabili in presenza di visi rivolti frontalmente e potrebbero non riconoscere (o farlo con un punteggio inferiore) visi oscurati
o se rivolti non frontalmente.
%Questa operazione (\textit{DetectFaces}) non salva alcuna informazione sul server.

\paragraph{Confronto volti} A differenza del metodo precedente, questo permette di misurare la probabilità che due volti siano la stessa persona; un punteggio associato ad ogni confronto
aiuta a valutarne il risultato.
Da un'immagine \textit{sorgente} contente un volto, questo viene confrontato con il volti presente nelle immagini \textit{destinazione}.
Anche in questo caso vengono forniture le coordinate spaziali di un immaginario rettangolo che circonda i visi che sono stati rilevati, assieme al grado di sicurezza che quel rettangolo contenga veramente un volto.
%Questa operazione (\textit{CompareFaces}) non salva alcuna informazione sul server.

\paragraph{Riconoscimento volti} Per trovare un volto all'interno di una collezione di immagini. Per prima cosa è necessaria la creazione di una collezione per il salvataggio dei volti, rappresentati come vettore di attributi.
Successivamente si fornisce al servizio un'immagine che provvederà alla ricerca di volti simili all'interno della collezione precedentemente creata.
Per ogni volto restituito viene associato, al solito, un livello di affidabilità la posizione del volto all'interno dell'immagine.


\subsection{Tariffe}
Il piano gratuito prevede ogni mese, per i primi 12 mesi, di:
\begin{itemize}
\item analizzare 5000 immagini,
\item memorizzare 1000 metadati facciali.
\end{itemize}
Altrimenti:
\begin{itemize}
\item per il primo milione di immagini\footnote{Ogni API che accetta una o più messaggi di input conta come un'immagine elaborata.}: $1$ dollaro ogni $1000$ immagini\footnote{Al mese.};
\item successivi 9 milioni di immagini: $0,80$ dollari ogni $1000$ immagini;
\item successivi 90 milioni di immagini: $0,60$ dollari ogni $1000$ immagini;
\item oltre i 100 milioni di immagini: $0,40$ dollari ogni $1000$ immagini.
\end{itemize}
Inoltre utilizzando le API per il riconoscimento dei volti, il servizio memorizza ogni volta la rappresentazione vettoriale dei volti. Questo comporta dei costi pari a $0,01$ dollari per 1000 metadati memorizzati al mese.  