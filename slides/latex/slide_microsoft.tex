% !TEX encoding = UTF-8
% !TEX program = pdflatex
% !TEX root = presentazione.tex
% !TeX spellcheck = it_IT
%
\begin{frame}[t]{Microsoft Computer Vision API}
\begin{itemize}
	\item Riconoscimento elementi dell'immagine
	\item Classificazione
	\item Riconoscimento volti
	\item Riconoscimento del testo
	\item Generazione di descrizioni
	\item Riconoscimento contenuti non adatti ai minori
	\item Altro:
	\begin{itemize}
		\item Creazione anteprime
		\item Identificazione tipo, colori e qualità immagine
		\item Estensione classificatore
	\end{itemize}
\end{itemize}
\end{frame}
%
%----------------------------------------------------------------------------------------
%	APPUNTI:
% 		- elementi come: oggetti, esseri viventi, azioni; oltre 2000
%			-ritorna un insieme di etichette (in formato JSON) che descrivono gli oggetti presenti nell’immagine
%		- classificazione secondo una tassonomia composta da 86 categorie
%		- identifica la posizione del volto e età/sesso
%		- testo presente nell'immagine e testo scritto a mano
%		- frase/i (in inglese) che descrive l'immagine
%		- le varie categorie comprendono gruppi per contenuti e non adatti a minori
%		- Estensione classificatore (Contenuto personalizzato):
%			- servirebbe appunto per estendere la tassonomia predefinita
%			- ad oggi è supportato solo: riconscimento personaggi famosi e oggetti di interesse (landmark)
%----------------------------------------------------------------------------------------
