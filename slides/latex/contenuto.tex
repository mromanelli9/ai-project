% !TEX encoding = UTF-8
% !TEX program = pdflatex
% !TEX root = presentazione.tex
% !TeX spellcheck = it_IT
%
% %%%%%%%%%%%%%%%%%%%%%%%%%%%%%%%%%%%%%%%
% APPUNTI:
%	-
%
%%%%%%%%%%%%%%%%%%%%%%%%%%%%%%%%%%%%%%%

\section{Contenuto}\label{sec:contenuto}
% %%%%%%%%%%%%%%%%%%%%%%%%%%%%%%%%%%%%%%%
%
% SLIDE 1: CONTESTO
%
% %%%%%%%%%%%%%%%%%%%%%%%%%%%%%%%%%%%%%%%
\begin{frame}[t]{Contesto}
\begin{itemize}
	\item Sicurezza ed autenticazione nel Web
	\item Le soluzioni in uso sono federate e centralizzate
	\begin{itemize}
		\item distinzione fra chi autentica, chi autorizza e chi offre il servizio
		\item Framework OAuth
		\item Protocollo OpenID
		\item etc.
	\end{itemize}
	\pause
	\vskip.5em%
	\item Problematiche: si affidano a terze parti ritenute fidate
	\begin{itemize}
		\item quanto possiamo fidarci?
		\item NSA Datagate, Snowden, etc.
	\end{itemize}
\end{itemize}
\end{frame}
% APPUNTI:
% OpenID è un meccanismo di identificazione creato da Brad Fitzpatrick di LiveJournal.
% Si tratta di un network distribuito e decentralizzato, nel quale la propria identità è un URL,
% e può essere verificata da qualunque server supporti il protocollo.
% -
% FOAF è un'ontologia per a descrivere persone, con le loro attività e le relazioni con altre persone e oggetti
