% !TEX encoding = UTF-8
% !TEX program = pdflatex
% !TEX root = presentazione.tex
% !TeX spellcheck = it_IT
%
\begin{frame}[t]{Servizi cognitivi}
\begin{itemize}
	\item Creare applicazioni in grado di analizzare e interpretare la realtà
	\item Aree principali:
	\begin{itemize}
		\item Visione
		\item Linguaggio
		\item Ricerca
		\item Conoscenza
	\end{itemize}
	\item A livello tecnico:
	\begin{itemize}
		\item chiamate REST (endpoint http)
		\item a cui passo dati (es. immagini) e parametri
	\end{itemize}
\end{itemize}
\end{frame}
%
%----------------------------------------------------------------------------------------
%	APPUNTI:
%		- realizzare applicazioni in grado di analizzare e interpretare la realtà
% 		- l’area di visione racchiude quelle tecniche per l’analisi, la rappresentazione di immagini e video
%		- area del lingaggio fornisce strumenti per capire meglio cosa vuole l’utente tramite l’interpretazione
%			della scrittura o del linguaggio naturale.
%----------------------------------------------------------------------------------------
