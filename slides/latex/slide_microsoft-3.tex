% !TEX encoding = UTF-8
% !TEX program = pdflatex
% !TEX root = presentazione.tex
% !TeX spellcheck = it_IT
%
\begin{frame}[t]{Microsoft Face API e Emotion API}
\begin{itemize}
	\item Rilevamento volti
	\item Confronto fra due volti
	\item Confronto/ricerca in un insieme di volti
	\item Emozioni: rabbia, paura, felicità, espressione neutra, tristezza, sorpresa, disprezzo e disgusto
\end{itemize}
\end{frame}
%
%----------------------------------------------------------------------------------------
%	APPUNTI:
% 		- più approfondita dei volti rispetto alle Computer Vision API
%			includendo anche funzioni di confronto e ricerca.
%		- Rileva i volti presenti nell’immagine (fino a 64) e restituisce le coordinate
%			del rettangolo che ingloba il viso, gli attributi facciali e i punti di riferimento.
%		- attributi facciali: eta, sesso, sorriso, tipologia di barba, la posizione tridimensionale
%			del volto (imbardata, rollio e beccheggio), la presenza di occhiali, le emozioni
%			emozioni: rabbia, paura, felicità, espressione neutra, tristezza, sorpresa, disprezzo e disgusto
%		- verifica volto: probabilità che due volti appartengano alla stessa persona
%		- identicificazione volti: identificare le persone sulla base di un volto e di un database di persone
%		- Ricerca volti simili: fornendo un volto obbiettivo e un insieme di candidati  all’interno del quale
%			eseguire la ricerca, questa funzione restituisce un piccolo insieme di volti che assomiglia al volto obiettivo.
%----------------------------------------------------------------------------------------
